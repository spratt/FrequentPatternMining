%%%%%%%%%%%%%%%%%%%%%%%%%%%%%%%%%%%%%%%%%%%%%%%%%%%%%%%%%%%%%%%%%%%%%%
% LatexTemplate
% November 10, 2012
% By Simon Pratt (mostly)
\documentclass[11pt]{article}

\usepackage{Assignment}
\usepackage{CGAlgorithms}
\usepackage{QuestionAnswer}
\usepackage{TheoremStuff}
\usepackage{HeaderStuff}
\usepackage{url}
\usepackage{hyperref}

%%%%%%%%%%%%%%%%%%%%%%%%%%%%%%%%%%%%%%%%%%%%%%%%%%%%%%%%%%%%%%%%%%%%%%
% Multicolumn stuff
\usepackage{multicol}
\usepackage{newclude} % \include without \clearpage

%%%%%%%%%%%%%%%%%%%%%%%%%%%%%%%%%%%%%%%%%%%%%%%%%%%%%%%%%%%%%%%%%%%%%%
% Configuration
\fancyhead[L]{COMP4106 - Final Project}
\fancyhead[R]{Simon Pratt - 100663987}
\bibliographystyle{plain}

%%%%%%%%%%%%%%%%%%%%%%%%%%%%%%%%%%%%%%%%%%%%%%%%%%%%%%%%%%%%%%%%%%%%%%
% Document
\begin{document}
\begin{multicols}{2} % remove line if you don't want multiple columns

% Content starts here

\section{The Problem}

The problem domain is to find frequent patterns in large data sets.

\subsection{Motivation}

Frequent patterns in data can give us insight into the semantic
meaning of the data.  In particular, if we are mining for frequent
patterns in a data set of actions we can identify typical workflow and
anomalous activity, which could be useful for security or process
optimization.

\section{Algorithms}

I implemented the following algorithms: \texttt{Apriori},
\texttt{FP-growth}, and \texttt{Eclat} \cite{Han2007,fpmlecture}.

\subsection{Apriori}

In which we describe the Apriori algorithm.

\subsection{FP-Growth}

In which we describe the FP-Growth algorithm.

\subsection{Eclat}

In which we describe the Eclat algorithm.

\section{Challenges}

In which we describe the challenges encountered.

\section{Results}
\label{sec:results}

In which we describe the results of the experiments.

\section{Future Work}

In which we describe possible enhancements to the study.

\appendix
\section{Running Requirements}

To run the code, you simply need to have \texttt{Python 2.6}
installed.  \texttt{Python 2.7} should also work, but the first line
in each python script references \texttt{Python 2.6} in particular,
which should be changed if running on \texttt{Python 2.7}.

There are 4 python scripts included in the code directory:
\texttt{dataset.py}, \texttt{fp\_mining.py}, \texttt{timing.py}, and
\texttt{unit\_tests.py}.

\texttt{dataset.py} simply handles the parsing of files into a simple
datastructure to use for pattern mining.  Running this file with the
name of a datafile as a parameter will run some simple tests on the
module and print the input dataset in its \texttt{Python} format.  The
expected file format is described in the next appendix.

\texttt{fp\_mining.py} containins the algorithm logic for
\texttt{Apriori}, \texttt{FP-Growth}, and \texttt{Eclat}.  The
expected usage of this script is as follows:

\texttt{fp\_mining.py [file] [k] [results]}

Where [file] is the name of an input file, [k] is the desired length
of patterns to mine, and [results] is an optional positive integer
specifying how many results to print.  If [results] is ommitted, all
resulting patterns will be printed.

Note also that this script produces detailed logging output to the
standard error stream on unix, which can be redirected by use of the
``2>'' operator.

\texttt{timing.py} runs the timed tests used to generate the output
discussed in the Results section.  This runs a sequence
of timed tests and prints aggregate information.

\texttt{unit\_tests.py} runs the unit tests for the code base.

\section{Datafile Format}

The datafiles used for testing were retrieved from the FIMI dataset
repository \cite{fimiDatasetRepo}.

% Content ends here

\bibliography{references}

\end{multicols} % remove line if you don't want multiple columns
\end{document}
